for Createing PWA (Prograssive Web App): 
---------------------------------------

1. Install the npm package
----------
use this command ("npm install next-pwa");

2. Ensure public/manifest.json exists
----------

Create public/manifest.json:

{
  "name": "An-Nahdah Online Institute",
  "short_name": "An-Nahdah",
  "description": "Blending authentic Islamic knowledge with practical skill development.",
  "lang": "en",
  "start_url": "/",
  "display": "standalone",
  "background_color": "#ffffff",
  "theme_color": "#1e40af",
  "icons": [
    {
      "src": "/icons/icon-192x192.png",
      "sizes": "192x192",
      "type": "image/png"
    },
    {
      "src": "/icons/icon-512x512.png",
      "sizes": "512x512",
      "type": "image/png"
    }
  ]
}


Make sure to put your app icons in public/icons/.


3. Update next.config.mjs for ESM
---------------------------------

import withPWA from "next-pwa";

/** @type {import('next').NextConfig} */
const nextConfig = {
  reactStrictMode: true,
  // any other Next.js config options here
};

export default withPWA({
  ...nextConfig,
  pwa: {
    dest: "public",
    register: true,
    skipWaiting: true,
    disable: process.env.NODE_ENV === "development",
  },
});


Step 4: Add meta tags in app/layout.js
--------------------------------------

export const metadata = {
  title: "An-Nahdah Online Institute",
  description: "Learn Islamic knowledge & skills online",
};

export default function RootLayout({ children }) {
  return (
    <html lang="en" suppressHydrationWarning>
      <head>
        <link rel="manifest" href="/manifest.json" />
        <meta name="theme-color" content="#1e3a8a" />
        <link rel="apple-touch-icon" href="/icons/icon-192x192.png" />
        <meta name="apple-mobile-web-app-capable" content="yes" />
        <meta name="mobile-web-app-capable" content="yes" />
      </head>
      <body className="antialiased">{children}</body>
    </html>
  );
}


## Requirements for the Install Button to Appear
------------------------------------------------

For Chrome to show the install button in the address bar (top right):

HTTPS: Your site must be served over HTTPS.

Manifest file: Your site must have a proper manifest.json file with at least:

name or short_name

start_url

icons (preferably 192x192 and 512x512 PNG)

display: standalone

Service Worker: You need a service worker registered that handles fetch events (so your app can work offline).

User engagement: Chrome usually waits until the user has visited the site at least twice or spent some time there before prompting.